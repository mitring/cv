\section{Владение прикладным инструментарием}
\cvitem{ОС}{Linux (RedHat-based), Windows, FreeBSD}
\cvitem{Языки}{PHP, Java, SQL, C, Delphi}
\cvitem{СУБД}{MySQL, PostgreSQL, MongoDB, HyTech, Oracle, DB2}
\cvitem{Управление версиями}{Subversion, Git}
\cvitem{Управление проектами}{Redmine, MediaWiki}
\cvitem{Компьютерная математика}{Octave, Scilab}
\cvitem{Верстка}{LaTeX, HTML, CSS}

\section{Ключевые навыки}
\cvlistitem{Проектирование, разработка, сопровождение программного обеспечения на всех этапах жизненного цикла --- от предпроектного обследования объекта автоматизации до гарантийного обслуживания разработанной системы и обучения пользователей.}
\cvlistitem{Системный подход к проектированию ПО.}
\cvlistitem{Максимальное использование потенциала свободного ПО для снижения стоимости разрабатываемых систем (с учетом лицензионных политик конкретных продуктов).}
\cvlistitem{Аккуратность и ответственность в работе с документами.}
\cvlistitem{Деловые отношения и ведение переговоров с заказчиками (в том числе государственными) и соисполнителями проектов.}
\cvlistitem{Подготовка докладов, выступление и проведение презентаций перед аудиторией, участие в конференциях и семинарах в качестве докладчика.}
\cvlistitem{Грамотная устная и письменная речь.}
\cvlistitem{Постоянное повышение квалификации, быстрая обучаемость.}
\cvlistitem{Организованность, пунктуальность, неконфликтность.}