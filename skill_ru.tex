\section{Инструментарий}
\cvitem{}{\textbf{Полужирным} отмечены инструменты, активно используемые в течение последнего года; обычным шрифтом --- активно использовавшиеся более года назад; \textit{курсивом} --- с которыми экспериментировал, но не использовал регулярно.}
\cvitem{ОС}{\textbf{Linux (RedHat-based)}, \textbf{Windows}, \textit{FreeBSD}}
\cvitem{Языки}{\textbf{Java}, \textbf{SQL}, Groovy, JavaScript, PHP, C, Delphi}
\cvitem{СУБД}{\textbf{PostgreSQL}, Oracle, MySQL, MongoDB, HyTech, \textit{DB2}}
\cvitem{FTS}{\textbf{ElasticSearch}, Sphinx}
\cvitem{Servers}{\textbf{Tomcat}, Weblogic, Apache HTTPD}
\cvitem{VCS}{\textbf{Git}, Subversion}
\cvitem{Build}{\textbf{Gradle}, \textbf{Maven}}
\cvitem{CI}{\textbf{Jenkins}, \textbf{Docker}, \textbf{Artifactory}, \textbf{Zabbix}, Nexus OSS}
\cvitem{Управление проектами}{\textbf{Jira}, \textbf{Confluence}, \textbf{Gitlab}, Redmine, MediaWiki}
\cvitem{Компьютерная математика}{Octave, Scilab}
\cvitem{Верстка}{\textbf{LaTeX}, \textbf{HTML}, \textbf{CSS}}

\section{Ключевые навыки}
\cvlistitem{Проектирование, разработка, сопровождение программного обеспечения на всех этапах жизненного цикла --- от предпроектного обследования объекта автоматизации до гарантийного обслуживания разработанной системы и обучения пользователей.}
\cvlistitem{Системный подход к проектированию ПО.}
\cvlistitem{Аккуратность и ответственность в работе с документами.}
\cvlistitem{Деловые отношения и ведение переговоров с заказчиками (в том числе государственными) и соисполнителями проектов.}
\cvlistitem{Подготовка докладов, выступление и проведение презентаций перед аудиторией, участие в конференциях и семинарах в качестве докладчика.}
\cvlistitem{Грамотная устная и письменная речь.}
\cvlistitem{Постоянное повышение квалификации, быстрая обучаемость.}
\cvlistitem{Организованность, пунктуальность, неконфликтность.}