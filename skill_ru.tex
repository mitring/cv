\section{Инструментарий}
\cvitem{}{\textbf{Полужирным} отмечено то, что активно используется в течение последнего года; обычным шрифтом --- что активно использовалось более года назад; \textit{курсивом} --- с чем экспериментировал, но не использовал регулярно}
\cvitem{ОС}{\textbf{Linux (RedHat-based)}, \textbf{Windows}, \textit{FreeBSD}}
\cvitem{Языки}{\textbf{Java}, \textbf{Groovy}, \textbf{JavaScript}, \textbf{SQL}, PHP, C, Delphi}
\cvitem{СУБД}{\textbf{Oracle}, MySQL, PostgreSQL, MongoDB, HyTech, \textit{DB2}}
\cvitem{VCS}{\textbf{Git}, Subversion}
\cvitem{Build}{\textbf{Gradle}, Maven}
\cvitem{CI}{\textbf{Jenkins}}
\cvitem{Управление проектами}{\textbf{Jira}, Redmine, MediaWiki}
\cvitem{Компьютерная математика}{Octave, Scilab}
\cvitem{Верстка}{\textbf{LaTeX}, \textbf{HTML}, \textbf{CSS}}

\section{Ключевые навыки}
\cvlistitem{Проектирование, разработка, сопровождение программного обеспечения на всех этапах жизненного цикла --- от предпроектного обследования объекта автоматизации до гарантийного обслуживания разработанной системы и обучения пользователей.}
\cvlistitem{Системный подход к проектированию ПО.}
\cvlistitem{Максимальное использование потенциала свободного ПО для снижения стоимости разрабатываемых систем (с учетом лицензионных политик конкретных продуктов).}
\cvlistitem{Аккуратность и ответственность в работе с документами.}
\cvlistitem{Деловые отношения и ведение переговоров с заказчиками (в том числе государственными) и соисполнителями проектов.}
\cvlistitem{Подготовка докладов, выступление и проведение презентаций перед аудиторией, участие в конференциях и семинарах в качестве докладчика.}
\cvlistitem{Грамотная устная и письменная речь.}
\cvlistitem{Постоянное повышение квалификации, быстрая обучаемость.}
\cvlistitem{Организованность, пунктуальность, неконфликтность.}