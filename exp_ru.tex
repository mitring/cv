\section{Опыт работы}
\cventry{10.2015--\\н. вр.~}{Ведущий программист (Senior Developer, Team Lead)}{\href{http://grfc.ru}{\underline{<<Главный радиочастотный центр>>}}, ФГУП}{Москва}{}{Телекоммуникации, связь\newline{}
\underline{Обязанности и достижения}:
\begin{itemize}
  \item Руководство разработкой \href{https://online.grfc.ru}{\underline{интернет-портала}} для организации электронного документооборота между ФГУП <<Главный радиочастотный центр>> и его контрагентами (Java Core; Java EE: EJB, JMS, JPA, CDI, Servlet, JAX-WS; HTML/CSS/JS; СУБД Oracle). Наиболее значимые достижения:
  \begin{itemize}
    \item успешно внедрил портал в промышленную эксплуатацию;
    \item интегрировал портал с \href{http://taxcom.ru/dokumentooborot/}{\underline{оператором электронного документооборота}};
   	\item спроектировал и разработал B2B web-сервисы для наиболее крупных пользователей радиочастотного спектра РФ.
  \end{itemize}
  \item Руководство разработкой \href{https://online.grfc.ru/sign.verify}{\underline{сервиса проверки электронных подписей}} (Java Core; Java EE: EJB, JMS, Servlet; HTML/CSS/JS). Наиболее значимые достижения:
  \begin{itemize}
	\item успешно внедрил сервис в промышленную эксплуатацию;
	\item интегрировал сервис с \href{https://e-trust.gosuslugi.ru/}{\underline{инфраструктурой Минкомсвязи}}.
  \end{itemize}
  \item Разработка ПО предприятия для организации электронного документооборота, обработки радиочастотных заявок, интеграции с программами расчета электромагнитной совместимости (Java Core: Swing; Groovy; Java EE: EJB, JMS, JPA, CDI; СУБД Oracle). Наиболее значимые достижения:
  \begin{itemize}
	\item разработал ключевые компоненты GUI клиентского приложения;
	\item разработал и реализовал методику и алгоритмы проверки радиочастотных заявок по решениям ГКРЧ.
  \end{itemize}
  \item Внедрение и управление процессом continuous integration (Git/Gitolite, Jenkins, Nexus OSS, Gradle).
\end{itemize}}

\cventry{10.2014--10.2015~}{Старший программист (Senior Developer)}{\href{http://grfc.ru}{\underline{<<Главный радиочастотный центр>>}}, ФГУП}{Москва}{}{Телекоммуникации, связь\newline{}
\underline{Обязанности и достижения}:
\begin{itemize}
  \item Разработка \href{https://online.grfc.ru}{\underline{интернет-портала}} для организации электронного документооборота между ФГУП <<Главный радиочастотный центр>> и его контрагентами (Java Core; Java EE: EJB, JMS, JPA, Servlet, JAX-WS; HTML/CSS/JS; СУБД Oracle). Наиболее значимые достижения:
  \begin{itemize}
	\item реализовал ключевую бизнес-логику приложения;
	\item внедрил BIRT для создания отчетных форм, что позволило значительно ускорить генерацию отчетов (по сравнению с шаблонами в формате docx);
	\item интегрировал портал с внутренними IT-системами предприятия.
  \end{itemize}
\end{itemize}}
\cventry{}{}{}{}{}{
\begin{itemize}
  \item Разработка \href{https://online.grfc.ru/sign.verify}{\underline{сервиса проверки электронных подписей}} (Java Core; Java EE: EJB, JMS, Servlet; HTML/CSS/JS). Наиболее значимые достижения:
  \begin{itemize}
	\item реализовал все проверки ЭП исключительно с помощью свободных компонентов (BouncyCastle);
	\item интегрировал сервис с остальными IT-системами предприятия.
  \end{itemize}
\end{itemize}}

\cventry{09.2008--10.2014~}{Программист 1-ой категории (Senior Developer, Team Lead)}{\href{http://openinfotech.ru}{\underline{<<Открытые Информационные Технологии>>}}, ООО}{Москва}{}{Информационные технологии: разработка и системная интеграция\newline{}
\underline{Обязанности и достижения}:
\begin{itemize}
  \item Участие в создании крупной единой интегрированной информационной системы <<Соцстрах>> для Фонда социального страхования РФ --- все регионы РФ, 25000 пользователей (Delphi; Java Core; СУБД HyTech; поисковая машина Sphinx). Наиболее значимые достижения:
  \begin{itemize}
    \item исполнял обязанности заместителя руководителя проекта: готовил обновления, вел переговоры с заказчиком, участвовал в постановке задач и приемо-сдаточных работах;
    \item полностью отвечал за направление, связанное со сдачей в ФСС РФ бухгалтерской отчетности (начиная от анализа предметной области и заканчивая обучением и поддержкой пользователей);
    \item спроектировал, разработал и успешно внедрил ПО для планирования расходов ФСС РФ на оздоровление детей;
    \item разрабатывал ПО для ведения камеральных и выездных проверок ревизорами ФСС РФ.
  \end{itemize}
  \item Участие в разработке \href{http://fz122.fss.ru}{\underline{интернет-портала ФСС РФ}} (PHP; HTML/CSS/JS; Java Core; СУБД HyTech; поисковая машина Sphinx). Наиболее значимые достижения:
  \begin{itemize}
    \item спроектировал, разработал и успешно внедрил полнотекстовый поиск по документам с помощью Sphinx;
	\item разрабатывал OLAP-отчеты (собственный движок) по всем сферам деятельности ФСС РФ (бухгалтерия, путевки, профилактика на вредных производствах, выездные и камеральные проверки и т.д.);
    \item адаптировал портал под высокие нагрузки (горизонтальное масштабирование, шардинг, партицирование, денормализация, оптимизация запросов).
  \end{itemize}
  \item Проектирование, разработка и внедрение \href{http://f4.fss.ru}{\underline{шлюза}} для сдачи бухгалтерской отчетности с ЭП (PHP; HTML/CSS/JS; СУБД HyTech). Наиболее значимые достижения:
  \begin{itemize}
    \item успешный запуск системы в промышленную эксплуатацию в крайне сжатые сроки: от проектирования до запуска прошло меньше двух месяцев;
    \item адаптация системы под высокие нагрузки (конвейерная обработка, асинхронное выполнение, горизонтальное масштабирование, шардинг, партицирование, денормализация, оптимизация запросов).
  \end{itemize}
  \item Разработка модуля взаимодействия SAP ERP и \href{http://www.openinfotech.ru/index.php?service=4&subs=11}{\underline{ПО для трансформации бухгалтерской отчетности}} из стандартов РСБУ в стандарты МСФО: создание необходимого для взаимодействия API на основе технологии SAP RFC (C; Delphi).	
  \item Проектирование, разработка и внедрение \href{http://www.openinfotech.ru/index.php?service=4&subs=10}{\underline{виртуального терминала}} для оплаты услуг банковскими картами (PHP; HTML/CSS/JS; C; СУБД MySQL/PostgreSQL/HyTech). Наиболее значимые достижения:
  \begin{itemize}
    \item разработка архитектуры ПО, толерантной к высоким нагрузкам (конвейерная обработка, асинхронное выполнение);
    \item интеграция терминала с системой платежей банковскими картами (Uniteller), основными платежными системами (E-port, Cyberplat, ОСМП), специализированными платежными системами заказчиков.
  \end{itemize}
\item Проектирование и разработка ПО для учета авторских и смежных прав на объекты интеллектуальной собственности (PHP; HTML/CSS/JS; СУБД HyTech и MongoDB).
\item Внедрение практики continuous integration (Jenkins).
\item Разработка документации (в том числе по ГОСТ 34.*, 19.*), конкурсных заявок по 94-ФЗ, ТЗ, приемо-сдаточных сопроводительных документов. Внедрил LaTeX для подготовки документации, что позволило отслеживать изменения в документах с помощью VCS и обеспечило повторямость сборок документации.
\end{itemize}}

\cventry{09.2007--07.2013~}{Ассистент (совмещение)}{Национальный исследовательский ядерный университет <<МИФИ>>, кафедра <<Информационные системы и технологии>>}{Москва}{}{Наука, образование\newline{}
\underline{Обязанности и достижения}:
\begin{itemize}
\item Преподавание курсов:
  \begin{itemize}
    \item <<Информатика (алгоритмы и структуры данных)>> (лекции и лабораторные работы);
    \item <<Языки программирования и методы трансляции>> (лекции и лабораторные работы);
    \item <<Методы оптимизации>> (лекции и лабораторные работы);
    \item <<Численные методы>> (лабораторные работы);
    \item <<Теория игр и исследование операций>> (лекции).
  \end{itemize}
  \item Руководство курсовыми и дипломными работами студентов.
  \item Участие в выполнении хоздоговорных работ с различными заказчиками как в роли соисполнителя, так и в роли генерального конструктора СЧ ОКР.
  \item Разработка программ, календарных планов и прочих методических документов (УМКД) по стандартам ГОС-2 и ФГОС-3.
  \item Исполнение обязанностей секретаря государственной экзаменационной комиссии.
\end{itemize}}

\cventry{10.2005--09.2008~}{Программист (Junior, Middle Developer)}{<<Новые информационные системы и технологии>>, ЗАО}{Москва}{}{Информационные технологии: разработка и системная интеграция\newline{}
\underline{Обязанности и достижения}:
\begin{itemize}
  \item Участие в создании крупной единой интегрированной информационной системы <<Соцстрах>> для Фонда социального страхования РФ (все регионы РФ, 25000 пользователей) в качестве разработчика программного обеспечения для приема бухгалтерской отчетности (Delphi; СУБД HyTech).
  \item Участие в разработке \href{http://fz122.fss.ru}{\underline{интернет-портала ФСС РФ}} (PHP; HTML/CSS/JS; СУБД HyTech). Наиболее значимые достижения:
  \begin{itemize}
    \item спроектировал, реализовал и успешно внедрил web-приложение для взаимодействия страхователей и ФСС РФ в части сдачи бухгалтерской отчетности (работа отмечена дипломом конференции <<Молодежь и наука>>, проводимой МИФИ в 2007 году);
    \item разрабатывал OLAP-отчеты (собственный движок) по всем сферам деятельности ФСС РФ (бухгалтерия, путевки, профилактика на вредных производствах, выездные и камеральные проверки и т.д.).
  \end{itemize}
  \item Разработка документации (в том числе по ГОСТ 34.*, 19.*), конкурсных заявок по 94-ФЗ, ТЗ, приемо-сдаточных сопроводительных документов.
\end{itemize}}