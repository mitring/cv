\section{Опыт работы}
\cventry{04.2021--\\н. вр.~}{Старший инженер-программист (Senior Developer)}{\href{https://bostongene.com}{\underline{<<БостонДжин>>}}, ООО}{Москва}{}{Разработка облачной платформы для подбора таргетной терапии рака исходя из уникальных генетических особенностей пациента.\newline{}
\underline{Обязанности и достижения}:
\begin{itemize}
	\item Разработка backend сервиса визуализации биоинформатических расчетов в виде отчета, на который опирается врач при выборе лечения пациента (Java Core; Spring Boot; JPA; PostgreSQL; Kubernetes; Grafana; ELK):
	\begin{itemize}
		\item разработал эффективный механизм поиска генов и их биомаркеров по различным критериям;
		\item оптимизировал объем используемой памяти и скорость загрузки биоинформатических расчетов.
	\end{itemize}
	\item Проведение code-review, участие в обнаружении и устранении архитектурных недостатков и проблем с производительностью.		
\end{itemize}}	

\cventry{01.2018--04.2021}{Ведущий программист (Senior Developer), архитектор ПО}{\href{https://givc.ru}{\underline{<<Главный информационно-вычислительный центр Минкультуры РФ>>}}, ФГБУ}{Москва}{}{Цифровая трансформация организаций, подведомственных Министерству культуры (музеи, библиотеки, архивы).\newline{}
\underline{Обязанности и достижения}:
\begin{itemize}
	\item Разработка полнотекстового поиска по сводному каталогу библиотек России (Java Core; Spring; PostgreSQL; ElasticSearch):
	\begin{itemize}
		\item перевел полнотекстовый поиск с PostgreSQL на ElasticSearch, что значительно увеличило скорость поисковых запросов;
		\item создал отказоустойчивый кластер ElasticSearch, обслуживающий инфраструктуру поиска.
	\end{itemize}
\end{itemize}}
\cventry{}{}{}{}{}{
\begin{itemize}
    \item Проектирование и разработка backend облачной автоматизированной библиотечной информационной системы <<Библио-21>> (Java Core; Kotlin; Spring; Vert.x; Quarkus; JPA; PostgreSQL; Kafka; ElasticSearch; Logstash; MinIO S3; RFID).
    \item Проектирование и разработка backend электронного интерактивного архива научно-проектной и исполнительной документации объектов культурного наследия России (Java Core; Spring; JPA; PostgreSQL; ElasticSearch; MinIO S3).
    \item Проектирование и разработка backend реестра объектов нематериального культурного наследия России (Java Core; Spring; JPA; PostgreSQL; ElasticSearch; MinIO S3).
    \item Проведение code-review проектов других разработчиков, участие в обнаружении и устранении архитектурных недостатков и проблем с производительностью.
	\item Внедрение и управление процессом continuous integration:
	\begin{itemize}
		\item перевел сборку проектов с TeamCity на Jenkins, что позволило сэкономить средства организации на покупку лицензий;
		\item организовал сборку приложений в виде Docker-образов и их публикацию в приватный Docker Registry;
		\item организовал мониторинг с помощью Zabbix, что позволило своевременно реагировать на инциденты с производительностью или доступностью приложений и оборудования.
	\end{itemize}
\end{itemize}}

\cventry{10.2014--01.2018}{Старший программист, ведущий программист (Senior Developer, Team Lead)}{\href{https://grfc.ru}{\underline{<<Главный радиочастотный центр>>}}, ФГУП}{Москва}{}{Автоматизация процессов расчета электромагнитной совместимости (ЭМС) радиоэлектронных средств связи.\newline{}
\underline{Обязанности и достижения}:
\begin{itemize}
	\item Разработка \href{https://online.grfc.ru}{\underline{интернет-портала}} для организации электронного документооборота между ФГУП <<Главный радиочастотный центр>> и его контрагентами (Java Core; Java EE: EJB, JMS, JPA, CDI, Servlet, JAX-WS; HTML/CSS/JS; СУБД Oracle):
	\begin{itemize}
  		\item реализовал ключевую бизнес-логику приложения;
  		\item внедрил BIRT для создания отчетных форм, что позволило значительно ускорить генерацию отчетов (по сравнению с шаблонами в формате docx);
  		\item интегрировал портал с внутренними IT-системами предприятия;
	    \item интегрировал портал с \href{http://taxcom.ru/dokumentooborot/}{\underline{оператором электронного документооборота}};
       	\item спроектировал и разработал B2B web-сервисы для наиболее крупных пользователей радиочастотного спектра РФ (МТС, Билайн, Мегафон, Теле2);
		\item успешно внедрил портал в промышленную эксплуатацию.
	\end{itemize}
    \item Разработка \href{https://online.grfc.ru/sign.verify}{\underline{сервиса проверки электронных подписей}} (Java Core; Java EE: EJB, JMS, Servlet; HTML/CSS/JS):
	\begin{itemize}
		\item реализовал все проверки ЭП исключительно с помощью свободных компонентов (BouncyCastle);
  		\item интегрировал сервис с другими IT-системами предприятия;
	  	\item интегрировал сервис с \href{https://e-trust.gosuslugi.ru/}{\underline{инфраструктурой Минкомсвязи}};
  		\item успешно внедрил сервис в промышленную эксплуатацию.
	\end{itemize}
	\item Разработка бэкофис-системы для организации электронного документооборота, обработки радиочастотных заявок, интеграции с программами расчета электромагнитной совместимости (Java Core; Swing; Groovy; Java EE: EJB, JMS, JPA, CDI; СУБД Oracle):
	\begin{itemize}
		\item разработал ключевые компоненты GUI клиентского приложения;
		\item разработал и реализовал методику и алгоритмы проверки радиочастотных заявок по решениям ГКРЧ.
	\end{itemize}
	\item Внедрение и управление процессом Continuous Integration (Git/Gitolite, Jenkins, Nexus OSS, Gradle).
\end{itemize}}

\cventry{10.2005--10.2014~}{Программист, программист 1-ой категории (Junior, Middle, Senior Developer, Team Lead)}{\href{http://nist.ru}{\underline{<<Новые информационные системы и технологии>>}}, ЗАО; \href{http://openinfotech.ru}{\underline{<<Открытые Информационные Технологии>>}}, ООО}{Москва}{}{Разработка и системная интеграция программного обеспечения.\newline{}
\underline{Обязанности и достижения}:
\begin{itemize}
	\item Участие в создании крупной единой интегрированной информационной системы <<Соцстрах>> для Фонда социального страхования РФ --- все регионы РФ, 25000 пользователей (Delphi; Java Core; СУБД HyTech; поисковая машина Sphinx):
\end{itemize}}
\cventry{}{}{}{}{}{
\begin{itemize}
	\item[]	
	\begin{itemize}
    	\item исполнял обязанности заместителя руководителя проекта: готовил обновления, вел переговоры с заказчиком, участвовал в постановке задач и приемо-сдаточных работах;
	    \item полностью отвечал за направление, связанное со сдачей в ФСС РФ бухгалтерской отчетности (начиная от анализа предметной области и заканчивая обучением и поддержкой пользователей);
	    \item спроектировал, разработал и успешно внедрил ПО для планирования расходов ФСС РФ на оздоровление детей;
    	\item разрабатывал ПО для ведения камеральных и выездных проверок ревизорами ФСС РФ.
	\end{itemize}
	\item Участие в разработке \href{http://fz122.fss.ru}{\underline{интернет-портала ФСС РФ}} (PHP; HTML/CSS/JS; Java Core; СУБД HyTech; поисковая машина Sphinx):
	\begin{itemize}
		\item спроектировал, реализовал и успешно внедрил web-приложение для взаимодействия страхователей и ФСС РФ в части сдачи бухгалтерской отчетности (работа отмечена дипломом конференции <<Молодежь и наука>>, проводимой МИФИ в 2007 году);
		\item разрабатывал OLAP-отчеты (собственный движок) по всем сферам деятельности ФСС РФ (бухгалтерия, путевки, профилактика на вредных производствах, выездные и камеральные проверки и т.д.);
		\item спроектировал, разработал и успешно внедрил полнотекстовый поиск по судебным документам ФСС РФ с помощью Sphinx;
		\item адаптировал портал под высокие нагрузки (горизонтальное масштабирование, шардинг, партицирование, денормализация, оптимизация запросов).
	\end{itemize}
	\item Проектирование, разработка и внедрение \href{http://f4.fss.ru}{\underline{шлюза}} для сдачи бухгалтерской отчетности с ЭП (PHP; HTML/CSS/JS; СУБД HyTech):
	\begin{itemize}
		\item успешный запуск системы в промышленную эксплуатацию в крайне сжатые сроки: от проектирования до запуска прошло меньше двух месяцев;
		\item адаптация системы под высокие нагрузки (конвейерная обработка, асинхронное выполнение, горизонтальное масштабирование, шардинг, партицирование, денормализация, оптимизация запросов).
	\end{itemize}
	\item Разработка модуля взаимодействия SAP ERP и ПО для трансформации бухгалтерской отчетности из стандартов РСБУ в стандарты МСФО на основе технологии SAP RFC (C; Delphi).	
	\item Проектирование, разработка и внедрение виртуального терминала для оплаты услуг банковскими картами (PHP; HTML/CSS/JS; C; СУБД MySQL/PostgreSQL/HyTech):
	\begin{itemize}
    	\item разработка архитектуры ПО, толерантной к высоким нагрузкам (конвейерная обработка, асинхронное выполнение);
	    \item интеграция терминала с системой платежей банковскими картами (Uniteller), основными платежными системами (E-port, Cyberplat, ОСМП), специализированными платежными системами заказчиков.
	\end{itemize}
	\item Проектирование и разработка ПО для учета авторских и смежных прав на объекты интеллектуальной собственности (PHP; HTML/CSS/JS; СУБД HyTech и MongoDB).
	\item Внедрение практики continuous integration (Jenkins).
	\item Разработка документации (в том числе по ГОСТ 34.*, 19.*), конкурсных заявок по 94-ФЗ, ТЗ, приемо-сдаточных сопроводительных документов. Внедрил LaTeX для подготовки документации, что позволило отслеживать изменения в документах с помощью VCS и обеспечило повторяемость сборок.
\end{itemize}}

\cventry{09.2007--07.2013~}{Ассистент (совмещение)}{\href{https://mephi.ru}{\underline{НИЯУ <<МИФИ>>}}, кафедра <<Информационные системы и технологии>>}{Москва}{}{Обеспечение образовательного процесса на кафедре, участие в выполнении хозяйственных договоров кафедры.\newline{}
\underline{Обязанности и достижения}:
\begin{itemize}
\item Преподавание курсов:
  \begin{itemize}
    \item <<Информатика (алгоритмы и структуры данных)>> (лекции и лабораторные работы);
    \item <<Языки программирования и методы трансляции>> (лекции и лабораторные работы);
    \item <<Методы оптимизации>> (лекции и лабораторные работы);
    \item <<Численные методы>> (лабораторные работы);
    \item <<Теория игр и исследование операций>> (лекции).
  \end{itemize}
  \item Руководство курсовыми и дипломными работами студентов.
  \item Участие в выполнении хоздоговорных работ с различными заказчиками как в роли соисполнителя, так и в роли генерального конструктора СЧ ОКР.
\end{itemize}}
\cventry{}{}{}{}{}{
\begin{itemize}
  \item Разработка программ, календарных планов и прочих методических документов (УМКД) по стандартам ГОС-2 и ФГОС-3.
  \item Исполнение обязанностей секретаря государственной экзаменационной комиссии.
\end{itemize}}