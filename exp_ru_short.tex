\section{Опыт работы}
\cventry{09.2008--\\н. вр.~}{Программист 1-ой категории (team lead, системный архитектор)}{\href{http://openinfotech.ru}{\underline{<<Открытые Информационные Технологии>>}}, ООО}{Москва}{}{Информационные технологии: разработка и системная интеграция\newline{}
\underline{Обязанности и достижения}:
\begin{itemize}
\item Участие в разработке крупной единой интегрированной информационной системы \href{http://rkn.gov.ru/it/register/?id=90538}{\underline{<<Соцстрах>>}} для Фонда социального страхования РФ (все регионы РФ, 30000 пользователей) в качестве системного архитектора и заместителя руководителя проекта.
\item Участие в разработке и адаптации под высокие нагрузки \href{http://fz122.fss.ru}{\underline{интернет-портала ФСС РФ}} в качестве системного архитектора полнотекстового поиска и ведущего разработчика OLAP-отчетов.
\item Участие в разработке и адаптации под высокие нагрузки \href{http://f4.fss.ru}{\underline{шлюза ФСС РФ}} для сдачи бухгалтерской отчетности с ЭЦП в качестве системного архитектора.
\item Ведущий разработчик модуля взаимодействия SAP ERP и \href{http://www.openinfotech.ru/index.php?service=4&subs=11}{\underline{ПО для трансформации бухгалтерской отчетности}} из стандартов РСБУ в стандарты МСФО.
\item Системный архитектор \href{http://www.openinfotech.ru/index.php?service=4&subs=10}{\underline{виртуального терминала}} для оплаты услуг банковскими картами.
\item Системный архитектор сервера приложений для учета авторских и смежных прав на объекты интеллектуальной собственности.
\item Разработка и написание документации (в том числе по ГОСТ 34.*, 19.*), конкурсных заявок по 94-ФЗ, ТЗ, приемо-сдаточных сопроводительных документов.
\item Разработка форматов обмена данными между различными информационными системами.
\end{itemize}}

\cventry{09.2007--07.2013~}{Ассистент (совмещение)}{Национальный исследовательский ядерный университет <<МИФИ>>, кафедра \href{http://kaf36.ru}{\underline{<<Информационные системы и технологии>>}}}{Москва}{}{Наука, образование\newline{}
\underline{Обязанности и достижения}:
\begin{itemize}
\item Преподавание курсов:
  \begin{itemize}
  \item <<Информатика (алгоритмы и структуры данных)>> (лекции и лабораторные работы);
  \item <<Языки программирования и методы трансляции>> (лекции и лабораторные работы);
  \item <<Методы оптимизации>> (лекции и лабораторные работы);
	\item <<Численные методы>> (лабораторные работы);
  \item <<Теория игр и исследование операций>> (лекции).
  \end{itemize}
\item Руководство курсовыми и дипломными работами студентов, руководство практикантами.
\item Участие в выполнении хоздоговорных работ с различными заказчиками как в роли соисполнителя, так и в роли генерального конструктора СЧ ОКР.
\item Разработка программ, календарных планов и прочих методических документов (УМКД) по стандартам ГОС-2 и ФГОС-3.
\item Исполнение обязанностей секретаря государственной экзаменационной комиссии.
\end{itemize}}
\cventry{}{}{}{}{}{
\begin{itemize}
\item Поддержание в работоспособном состоянии информационной инфраструктуры кафедры (администрирование компьютерных классов и машин ППС).
\end{itemize}}

\cventry{10.2005--09.2008~}{Программист (junior , senior developer)}{<<Новые информационные системы и технологии>>, ЗАО}{Москва}{}{Информационные технологии: разработка и системная интеграция\newline{}
\underline{Обязанности и достижения}:
\begin{itemize}
\item Участие в разработке крупной единой интегрированной информационной системы \href{http://rkn.gov.ru/it/register/?id=90538}{\underline{<<Соцстрах>>}} для Фонда социального страхования РФ (все регионы РФ, 30000 пользователей) в качестве ведущего разработчика.
\item Участие в разработке \href{http://fz122.fss.ru}{\underline{интернет-портала ФСС РФ}} в качестве системного архитектора приложения для взаимодействия страхователей и ФСС РФ в части сдачи бухгалтерской отчетности и ведущего разработчика OLAP-отчетов.
\item Разработка и написание документации (в том числе по ГОСТ 34.*, 19.*), конкурсных заявок по 94-ФЗ, ТЗ, приемо-сдаточных сопроводительных документов.
\end{itemize}}