\section{Опыт работы}
\cventry{10.2015--\\н. вр.~}{Ведущий программист (Team Lead)}{\href{http://grfc.ru}{\underline{<<Главный радиочастотный центр>>}}, ФГУП}{Москва}{}{Информационные технологии: разработка и системная интеграция\newline{}
\underline{Обязанности и достижения}:
\begin{itemize}
\item Руководство разработкой \href{https://online.grfc.ru}{\underline{интернет-портала}} для организации электронного документооборота между ФГУП <<Главный радиочастотный центр>> и его контрагентами; интеграцией портала с внутренними IT-системами предприятия; интеграцией портала с \href{https://sbis.ru/edo}{\underline{оператором ЭДО}} (Java Core; Java EE: EJB, JMS, JPA, Servlet, JAX-WS, JAX-RS; HTML/CSS/JS; СУБД Oracle).
\item Руководство разработкой \href{https://online.grfc.ru/sign.verify}{\underline{сервиса проверки электронных подписей}}; интеграцией сервиса с остальными IT-системами предприятия и \href{https://e-trust.gosuslugi.ru/}{\underline{инфраструктурой Минкомсвязи}} (Java Core; Java EE: EJB, JMS, Servlet, JAX-WS, JAX-RS; HTML/CSS/JS).
\item Разработка внутренней системы предприятия для организации электронного документооборота и интеграции с программами для расчета электромагнитной совместимости (Java Core: Swing; Java EE: EJB, JMS, JPA).
\item Управление процессом continuous integration (Git/Gitolite, Jenkins, Gradle).
\end{itemize}}

\cventry{10.2014--10.2015~}{Старший программист (Senior Developer)}{\href{http://grfc.ru}{\underline{<<Главный радиочастотный центр>>}}, ФГУП}{Москва}{}{Информационные технологии: разработка и системная интеграция\newline{}
\underline{Обязанности и достижения}:
\begin{itemize}
\item Разработка \href{https://online.grfc.ru}{\underline{интернет-портала}} для организации электронного документооборота между ФГУП <<Главный радиочастотный центр>> и его контрагентами; интеграция портала с внутренними IT-системами предприятия (Java Core; Java EE: EJB, JMS, JPA, Servlet, JAX-WS, JAX-RS; HTML/CSS/JS; СУБД Oracle).
\item Разработка \href{https://online.grfc.ru/sign.verify}{\underline{сервиса проверки электронных подписей}}; интеграция сервиса с остальными IT-системами предприятия и \href{https://e-trust.gosuslugi.ru/}{\underline{инфраструктурой Минкомсвязи}} (Java Core; Java EE: EJB, JMS, Servlet, JAX-WS, JAX-RS; HTML/CSS/JS).
\end{itemize}}

\cventry{09.2008--10.2014~}{Программист 1-ой категории (Senior Developer, Team Lead)}{\href{http://openinfotech.ru}{\underline{<<Открытые Информационные Технологии>>}}, ООО}{Москва}{}{Информационные технологии: разработка и системная интеграция\newline{}
\underline{Обязанности и достижения}:
}
\cventry{}{}{}{}{}{
\begin{itemize}
\item Участие в разработке крупной единой интегрированной информационной системы \href{http://rkn.gov.ru/it/register/?id=101637}{\underline{<<Соцстрах>>}} для Фонда социального страхования РФ --- все регионы РФ, 25000 пользователей (Delphi; СУБД HyTech; поисковая машина Sphinx):
  \begin{itemize}
  \item заместитель руководителя проекта: подготовка обновлений, переговоры с заказчиком, участие в постановке задач и приемо-сдаточных работах;
  \item системный архитектор программного обеспечения для приема бухгалтерской отчетности: анализ предметной области и разработка архитектуры ПО; создание логической и физической структуры БД, серверной части ПО; разработка алгоритмов автоматизации и клиентской части ПО; обучение (командировки в учебные центры ФСС РФ) и поддержка пользователей (e-mail, форум, телефон);
  \item системный архитектор программного обеспечения для планирования расходов на оздоровление детей: список обязанностей в предыдущем подпункте;
  \item разработчик программного обеспечения для ведения камеральных и выездных проверок: разработка логической структуры БД, разработка алгоритмов автоматизации и клиентской части ПО.
  \end{itemize}
\item Участие в разработке \href{http://fz122.fss.ru}{\underline{интернет-портала ФСС РФ}} (PHP; HTML/CSS/JS; Java Core; СУБД HyTech; поисковая машина Sphinx):
  \begin{itemize}
  \item системный архитектор сервиса полнотекстового поиска: разработка логической и физической структуры БД, средств и регламента взаимодействия СУБД и поисковой машины Sphinx, парсера документов наиболее распространенных форматов (PDF, DOC/DOCX, RTF, ODT, HTML);
	\item ведущий разработчик OLAP-отчетов по всем сферам деятельности ФСС РФ (бухгалтерия, путевки, профилактика на вредных производствах, выездные и камеральные проверки и т.д.);
  \item адаптация системы под высокие нагрузки (горизонтальный и вертикальный шардинг БД, оптимизация запросов).
  \end{itemize}
\item Участие в разработке \href{http://f4.fss.ru}{\underline{шлюза ФСС РФ}} для сдачи бухгалтерской отчетности с ЭЦП в качестве системного архитектора (PHP; HTML/CSS/JS; СУБД HyTech):
  \begin{itemize}
  \item анализ предметной области и разработка архитектуры ПО;
  \item разработка форматов обмена данными между прикладными программами для подготовки отчетности и шлюзом (XML, XSD);
  \item создание логической и физической структуры БД;
  \item разработка алгоритмов автоматизации и клиентской части ПО;
  \item адаптация системы под высокие нагрузки (горизонтальный и вертикальный шардинг БД, оптимизация запросов).
  \end{itemize}
\item Ведущий разработчик модуля взаимодействия SAP ERP и \href{http://www.openinfotech.ru/index.php?service=4&subs=11}{\underline{ПО для трансформации бухгалтерской отчетности}} из стандартов РСБУ в стандарты МСФО: создание необходимого для взаимодействия API на основе технологии SAP RFC (C; Delphi).	
\item Системный архитектор \href{http://www.openinfotech.ru/index.php?service=4&subs=10}{\underline{виртуального терминала}} для оплаты услуг банковскими картами (PHP; HTML/CSS/JS; C; СУБД MySQL/PostgreSQL/HyTech):
  \begin{itemize}
  \item анализ предметной области и разработка архитектуры ПО;
  \item интеграция терминала с системой платежей банковскими картами (Uniteller), основными платежными системами (E-port, Cyberplat, ОСМП), специализированными платежными системами заказчиков;
  \item создание логической и физической структуры БД;
  \item адаптация системы под высокие нагрузки (помещение запросов в очереди, асинхронная обработка).
  \end{itemize}
\item Системный архитектор сервера приложений для учета авторских и смежных прав на объекты интеллектуальной собственности (PHP; HTML/CSS/JS; СУБД HyTech и MongoDB):
  \begin{itemize}
  \item анализ предметной области и разработка архитектуры ПО;
  \item создание логической и физической структуры БД;
  \item создание REST API для обработки данных.
  \end{itemize}	
\item Разработка и написание документации (в том числе по ГОСТ 34.*, 19.*), конкурсных заявок по 94-ФЗ, ТЗ, приемо-сдаточных сопроводительных документов.
\item Разработка форматов обмена данными между различными информационными системами.
\end{itemize}}

\cventry{09.2007--07.2013~}{Ассистент (совмещение)}{Национальный исследовательский ядерный университет <<МИФИ>>, кафедра <<Информационные системы и технологии>>}{Москва}{}{Наука, образование\newline{}
\underline{Обязанности и достижения}:
\begin{itemize}
\item Преподавание курсов:
  \begin{itemize}
  \item <<Информатика (алгоритмы и структуры данных)>> (лекции и лабораторные работы);
  \item <<Языки программирования и методы трансляции>> (лекции и лабораторные работы);
  \item <<Методы оптимизации>> (лекции и лабораторные работы);
  \item <<Численные методы>> (лабораторные работы);
  \item <<Теория игр и исследование операций>> (лекции).
  \end{itemize}
\item Руководство курсовыми и дипломными работами студентов, руководство практикантами.
\item Участие в выполнении хоздоговорных работ с различными заказчиками как в роли соисполнителя, так и в роли генерального конструктора СЧ ОКР.
\end{itemize}}
\cventry{}{}{}{}{}{
\begin{itemize}
\item Разработка программ, календарных планов и прочих методических документов (УМКД) по стандартам ГОС-2 и ФГОС-3.
\item Исполнение обязанностей секретаря государственной экзаменационной комиссии.
\item Поддержание в работоспособном состоянии информационной инфраструктуры кафедры (администрирование компьютерных классов и машин ППС).
\end{itemize}}

\cventry{10.2005--09.2008~}{Программист (Junior, Senior developer)}{<<Новые информационные системы и технологии>>, ЗАО}{Москва}{}{Информационные технологии: разработка и системная интеграция\newline{}
\underline{Обязанности и достижения}:
\begin{itemize}
\item Участие в разработке крупной единой интегрированной информационной системы \href{http://rkn.gov.ru/it/register/?id=101637}{\underline{<<Соцстрах>>}} для Фонда социального страхования РФ (все регионы РФ, 25000 пользователей) в качестве ведущего разработчика программного обеспечения для приема бухгалтерской отчетности (Delphi; СУБД HyTech): создание логической и физической структуры БД; разработка клиентской части ПО; обучение (командировки в учебные центры ФСС РФ) и поддержка пользователей (e-mail, форум, телефон).
\item Участие в разработке \href{http://fz122.fss.ru}{\underline{интернет-портала ФСС РФ}} (PHP; HTML/CSS/JS; СУБД HyTech):
  \begin{itemize}
  \item системный архитектор приложения для взаимодействия страхователей и ФСС РФ в части сдачи бухгалтерской отчетности (работа отмечена дипломом конференции <<Молодежь и наука>>, проводимой МИФИ в 2007-ом году);
  \item ведущий разработчик OLAP-отчетов по всем сферам деятельности ФСС РФ (бухгалтерия, путевки, профилактика на вредных производствах, выездные и камеральные проверки и т.д.).
  \end{itemize}
\item Разработка и написание документации (в том числе по ГОСТ 34.*, 19.*), конкурсных заявок по 94-ФЗ, ТЗ, приемо-сдаточных сопроводительных документов.
\end{itemize}}